% !TeX encoding = UTF-8
% !TeX spellcheck = es_ES
% !TeX root = Proto.tex
%!TEX root=Proto.tex
Como base de nuestro Bus, vamos a utilizar I2C puesto que prácticamente todo los micro-controladores tienen
hardware dedicado para dicho protocolo y es el que menos pines nos requieren (2). SPI se descarta por requerir
mas pines (3+D). Dicho esto se ha evaluado y descartado los Buses CAN y RS-XXX por requerir como mínimo un chip
más (Transceivers o PHY) y no tendiendo todos los micros soporte a dichos buses. Estos buses, aun estando
pensados para realizar comunicaciones externas, es posible usarlos en cortas distancias, como una placa o entre
placas de un dispositivo cerrado.

Basandonos en el Modelo OSI de redes definiremos la capa física, I2C se encargara del transporte y la
comunicacion de los datos. Luego nos queda las capas superiores (sesión-aplicación) donde definiremos un
protocolo propio, uno base para poder gestionar los dispositivos y varios por encima para cada tipo de panel
que queramos hacer.

\subsection{Capa Fisica}
I2C ya nos marca unos mínimos de como deben ser las conexiones, en este apartado dejaremos simplemente la
definición de los conectores y los cables a usar.

La conexión física es en estrella, partiendo del nodo central o maestro. Pero todos los dispositivos compartiran
las lineas I2C (SDA y SCL)

Los cables serán AWG 22 e irán a un conector JST\_PH2.0 de 5 pines:
\begin{itemize}
    \item VCC (3.3v 100mA max)
    \item SDA (PullUp 1K Master)
    \item SCL (PullUp 1K Master)
    \item GND
    \item INT (PullUp 1K Master)
\end{itemize}


\subsection{Protocolo Sesion}
Este protocolo sera el base que todos los dispositivos deben implementar, tendrá funciones de
\begin{itemize}
    \item Init
    \item Reset
    \item Identificacion
    \item Gestion de direcciones I2C
    \item \dots
\end{itemize}

Por una parte se definirán el flujo para iniciar los dispositivos y una serie de Registros y Operaciones I2C para
el funcionamiento mínimo de los dispostivos

\subsection{Protocolos de Aplicacion}

Por encima de I2C y de la sesión se definirán otros Registros/Operaciones I2C para cada uno de los tipos de
dispositivos que existan,  por ejemplo para botones, TFTs, palancas de control,\dots.
