% !TeX encoding = UTF-8
% !TeX spellcheck = es_ES
% !TeX root = Proto.tex
%!TEX root=Proto.tex

Como base de nuestro Bus, vamos a utilizar I2C puesto que prácticamente todo los micro-controladores tienen % chktex 8
hardware dedicado para dicho protocolo y es el que menos pines nos requieren (2). SPI se descarta por requerir
mas pines (3+D). Dicho esto se ha evaluado y descartado los Buses CAN y RS-XXX por requerir como mínimo un chip % chktex 8
más (Transceivers o PHY) y no tendiendo todos los micros soporte a dichos buses. Estos buses, aun estando
pensados para realizar comunicaciones externas, es posible usarlos en cortas distancias, como una placa o entre
placas de un dispositivo cerrado.

Basandonos en el Modelo OSI de redes definiremos la capa física, I2C se encargara del transporte y la
comunicacion de los datos. Luego nos queda las capas superiores (sesión-aplicación) donde definiremos un % chktex 8
protocolo propio, uno base para poder gestionar los dispositivos y varios por encima para cada tipo de panel
que queramos hacer.

\subsection{Capa Fisica}
I2C ya nos marca unos mínimos de como deben ser las conexiones, en este apartado dejaremos simplemente la
definición de los conectores y los cables a usar.

La conexión física es en estrella, partiendo del nodo central o maestro. Pero todos los dispositivos compartiran
las lineas I2C (SDA y SCL)

Los cables serán AWG 22 e irán a un conector JST\_PH2.0 de 5 pines:

\begin{itemize}
    \item{} VCC (3.3v 100mA max)
    \item{} SDA (PullUp 1K Master)
    \item{} SCL (PullUp 1K Master)
    \item{} GND
    \item{} INT (PullUp 1K Master)
\end{itemize}


\subsection{Capa enlace}
En el caso de I2C los microcontroladores contienen hardware para gestionar el bus. En concreto el que hace de
master I2C. Este microcontrolador toma el control de las lineas I2C, tanto de SDA como SCL, (siendo esta ultima
el reloj para mandar y recivir datos). Como el Master controla el reloj, este debe saber cuando bytes va recibir.
Y el slave debe devolver ese mismo numero de bytes (sino los micro se bloquean)

Una trama fisica I2C tipica tiene la forma (Negro Controla Master, Rojo controla Slave):
\begin{figure}[H]
    \centering{}
    \begin{tikztimingtable}[
            scale=0.69,
            timing/dslope=0.1,
            timing/.style={x=4ex,y=2ex},
            x=1ex,
            timing/rowdist=5ex,
            timing/name/.style={font=\sffamily\scriptsize}
        ]
        SDA & L D{A6} D{A5} D{A4} D{A3} D{A2} D{A1} D{A0} D{R/W} D{[red]ACK} 8D{CMD} 8D{[red]Response}U \\
        SCL & h L 49{c} U \\
        \extracode()
        \begin{pgfonlayer}{background}
            \begin{scope}[semitransparent,semithick]
                \vertlines[darkgray,dotted]{2,6,...,106} % chktex 11
            \end{scope}
        \end{pgfonlayer}
    \end{tikztimingtable}
    \caption{Trama I2C de ejemplo}
    \label{fig:I2C Example}
\end{figure}

El bit R/W se supone que es para indicar si una operacion es de escritura o de lectura. Como en las librerias de
Arduino no se nos permite controlar este bit, no lo usaremos en el protocolo.

En codigo una peticion de un comando (CMD) que devuele una cantidad de bytes (SIZE) debe hacerse como:

\begin{mdCode}
    \lstset{language=C++,style=cppstyle}
    \begin{lstlisting}
Wire.beginTransmission(address);    // Iniciar comunicacion
Wire.write(CMD);                    // Enviar Commando
Wire.endTransmission(false);        // Preparar para response
Wire.requestFrom(address,SIZE);     // Solicitar response
while(Wire.available()>0){          // Tantas veces como SIZE
    byte slaveByte2 = Wire.read();  // Leer byte.
}
Wire.endTransmission(true);        // Liberar Bus
\end{lstlisting}
\end{mdCode}

Para nuestro protocolo base nos definiremos una trama de Request y una de Response. Ambas tramas constaran
de un byte cabecera incial y de otro final de CRC.\@

La cabecera en ambos casos contendra 6 bits para indicar un campo de Tamaño. Y los dos bits restantes para indicar
estados rapidos.

Por simplificar la funcion CRC sera la suma de todos los bytes haciendo un ``xor'' con la cadena  ``0xAA''.\@


\subsubsection{Trama Request}
La trama de request consta de 4 bytes, mas todos los bytes que el comando necesite para los parametros.

\begin{figure}[H]
    \centering{}
    \begin{bytefield}[bitformatting={\small\bfseries},
            bitwidth=2em,endianness=big]{8}
        \bitheader[]{0-7} \\ % chktex 8
        \begin{leftwordgroup}[]{Header}
            \bitbox{1}{\small Ext} & \bitbox{1}{\small R/W} & \bitbox{6}{R\_Size} % chktex 1
        \end{leftwordgroup} \\
        \bitbox{8}{Proto ID*} \\
        \bitbox{8}{CMD}\\
        \begin{leftwordgroup}[]{Parametros}
            \bitbox{8}{Params}\\
            \bitbox{8}{\dots}
        \end{leftwordgroup} \\
        \bitbox{8}{CRC}\\
    \end{bytefield}
    \caption{Trama Request}
    \label{fig:Request}
\end{figure}
Campos:
\begin{itemize}
    \item{} \textbf{Header}: cabecera de la peticion, con un par de flags y el tamaño de Respuesta.
          \begin{itemize}
              \item{} \textbf{Ext}: 1 Bit, Si esta habilitado la request sera de un Protocolo Extendido. Se debe
                    incluir el campo \textbf{Proto ID}
              \item{} \textbf{R/W}: 1 Bit, Identificador de tipo de request, ``Read'' o ``Write''. Como no podemos
                    controlar el bit I2C, lo ponemos aqui para asi poder indicarlo aqui. Asi nos permite utilizar estas tramas
                    sobre otros transportes.
              \item{} \textbf{R\_Size}: 6 Bits, Tamaño de la respuesta. Indica al cliente el numero de bytes que va a requerir el
                    Master. De esta forma el Dispositivo sabe cuantos bytes enviar. Al usar 6 bits nos permite indicar hasta 63 Bytes.
                    Se intentara indicar el numero exacto que la respuesta generara (\textbf{SHOULD} segun terminologia RFC)
          \end{itemize}
    \item{} \textbf{Proto ID}: Identificador del Protocolo de aplicacion, si el bit \textbf{Ext} es 0, la peticion
          es para un protocolo basico y este campo no debe enviarse.
    \item{} \textbf{CMD}: El identificador del Comando Requerido.
    \item{} \textbf{Params}: Los bytes que requiera el comando como parametros.
    \item{} \textbf{CRC}: Byte con el resultado de la funcion CRC segun se ha definido antes.
\end{itemize}

\subsubsection{Trama Response}
La respuesta a una request debe ser del mismo numero de bytes indicados en el campo \textbf{R\_Size} de
la misma, indistitanmente si los necesita o no. Para los casos que la respuesta no los necesite se debera incluir
tantos bytes de PAD como sea necesario para llenar el hueco. \textbf{R\_Size} incluye la cabecera y el CRC.\@
\begin{figure}[H]
    \centering{}
    \begin{bytefield}[bitformatting={\small\bfseries},
            bitwidth=2em,endianness=big]{8}
        \bitheader[]{0-7} \\ % chktex 8
        \begin{leftwordgroup}[]{R\_Size}
            \bitbox{1}[]{\small OF} & \bitbox{1}{\small Err} & \bitbox{6}{Size} \\ % chktex 1
            \begin{rightwordgroup}[]{Size}
                \bitbox{8}{Response} \\
                \bitbox{8}{\dots}
            \end{rightwordgroup}\\
            \bitbox{8}{PAD*} \\
            \bitbox{8}{\dots} \\
            \bitbox{8}{CRC}
        \end{leftwordgroup}
    \end{bytefield}
    \caption{Trama Response}
    \label{fig:Response}
\end{figure}

Campos:
\begin{itemize}
    \item{} \textbf{Header}: cabecera de la peticion, con un par de flags y el tamaño de Respuesta.
          \begin{itemize}
              \item{} \textbf{OF}: 1 Bit, Indica si los datos a responder no caben en tamaño asignado por \textbf{R\_Size}.
              \item{} \textbf{Err}: 1 Bit,  Indica si ha habido algun error. En caso afirmativo la respuesta sera un codigo de error
              \item{} \textbf{Size}: 6 Bits, Tamaño de la respuesta. Indica al Master cuantos bytes son realmente utiles, el resto seran
                    relleno
          \end{itemize}
    \item{} \textbf{Response}: Los Bytes con la respuesta. Si hay un error bit \textbf{Err} 1, sera el codigo de error.
    \item{} \textbf{PAD}: Bytes de relleno.
    \item{} \textbf{CRC}: Byte con el resultado de la funcion CRC segun se ha definido antes.
\end{itemize}

En el caso de que el tamaño solicitado sea insufciente para la respuesta, se debe crear una trama con los bits \textbf{OF} y \textbf{Err}
a 1 y la respuesta tendra codigo de error \textbf{REQUEST\_SIZE\_ERROR}\sidenote{Sea cual sea el valor
    definido, Mirar el anexo correspondiente}.

No obstante se puede dar el caso de que el dispositivo pueda responder parcialmente, ya que en el protocolo se tiene en cuenta el envio
y procesado de listas de elementos. El caso de que el cliente aun pueda responder con al menos un elemento la repuesta contendra el bit
\textbf{OF} a 1 pero \textbf{Err} a 0, esto es una indicacion de que se ha podido devolver parte de la informacion.

Como por ejemplo la la lista eventos del usuario. Se puede dar el caso de que el dispositivo tenga acumulados 10 eventos, pero la Request
de obtencion de eventos solo tenga hueco para 5. En este caso \textbf{Err} estara 0 y la response solo contendra 5 Eventos. El master
tendra que realizar otra request para obtener los 5 restantes.

El relleno o \textbf{PAD} son bytes sin informacion util pero si entran en el calculo del CRC.\@ Existen diferentes funciones para poner
un relleno, como un valor fijo, un patron conocido (FF00 \dots{} 00FF), \dots{}. Para este protocolo solo se exigira que todos los bytes
tengan el mismo valor y se recomienda que el valor sea el numero de bytes con relleno. En un futuro esto se podra comprobar.

\subsection{Protocolo Sesion}
Este protocolo sera el base que todos los dispositivos deben implementar, tendra funciones de
\begin{itemize}
    \item{} Init
    \item{} Reset
    \item{} Identificacion
    \item{} Gestion de direcciones I2C
    \item{} \dots{}
\end{itemize}

Por una parte se definiran el flujo para iniciar los dispositivos y una serie de Registros y Operaciones I2C para
el funcionamiento minimo de los dispostivos

\subsection{Protocolos de Aplicacion}

Por encima de I2C y de la sesion se definiran otros Registros/Operaciones I2C para cada uno de los tipos de
dispositivos que existan,  por ejemplo para botones, TFTs, palancas de control,\dots.
